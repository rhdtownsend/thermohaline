\documentclass{article}

\usepackage{amsmath}
\usepackage{xcolor}

%% Page dimensions

\special{papersize=8.5in,11in}
\setlength{\pdfpageheight}{\paperheight}
\setlength{\pdfpagewidth}{\paperwidth}

\setlength{\textwidth}{6.5in} 
\setlength{\textheight}{9in}
\setlength{\topmargin}{0in} 
\setlength{\oddsidemargin}{0in}
\setlength{\evensidemargin}{0in} 
\setlength{\headheight}{0in}
\setlength{\headsep}{0in} 
\setlength{\hoffset}{0in}
\setlength{\voffset}{0in}

%% Macros

\newcommand{\shat}{\hat{\sigma}}

\newcommand{\phat}{\hat{\psi}}
\newcommand{\That}{\hat{T}}
\newcommand{\Chat}{\hat{C}}
\newcommand{\Ahat}{\hat{A}}

\newcommand{\ptil}{\tilde{\psi}}
\newcommand{\Ttil}{\tilde{T}}
\newcommand{\Ctil}{\tilde{C}}
\newcommand{\Atil}{\tilde{A}}

\newcommand{\khat}{\hat{k}}
\newcommand{\lhatf}{\hat{l}_{\rm f}}

\newcommand{\Prd}{\mathrm{Pr}}
\newcommand{\Rz}{R_{0}}
\newcommand{\HB}{H_{B}}
\newcommand{\DB}{D_{B}}
\newcommand{\EhatT}{\hat{E}_{T}}
\newcommand{\EhatC}{\hat{E}_{C}}
\newcommand{\Ehatp}{\hat{E}_{\psi}}

\newcommand{\Amat}{\boldsymbol{\mathsf{A}}}
\newcommand{\Bmat}{\boldsymbol{\mathsf{B}}}
\newcommand{\uvec}{\mathbf{u}}
\newcommand{\vvec}{\mathbf{v}}

\newcommand{\ii}{\mathrm{i}}

\begin{document}

\section{Linearized Equations}

As the starting point, let's first restate the linearized equations
given in Adrian's notes:
%
\begin{align}
  \begin{split}
  \shat \phat_{m} &=
  \ii \frac{\lhatf \khat_z}{\khat^2_m} \Ehatp \left[
    \left( \lhatf^2 - \khat^2_{m+1} \right) \phat_{m+1}
    + \left( \lhatf^2 - \khat^2_{m-1} \right) \phat_{m-1}
    \right]
  - \Prd \khat^2_m \phat_m
  - \ii \Prd \frac{m \lhatf}{\khat_m^2} \left( \That_m - \Chat_m \right) \\
  & \qquad + \ii \HB \khat_z \Ahat_m,
  \end{split} \\
  \shat \That_m &=
  - \ii \lhatf \khat_z \Ehatp \left( \That_{m+1} + \That_{m-1} \right)
  - \lhatf \khat_z \EhatT \left( \phat_{m-1} - \phat_{m+1} \right)
  - \ii m \lhatf \phat_m
  - \khat^2_m \That_m, \\
  \shat \Chat_m &=
  - \ii \lhatf \khat_z \Ehatp \left( \Chat_{m+1} + \Chat_{m-1} \right)
  - \lhatf \khat_z \EhatC \left( \phat_{m-1} - \phat_{m+1} \right)
  - \ii m \lhatf \frac{\phat_m}{\Rz}
  - \tau \khat^2_m \Chat_m, \\
  \shat \Ahat_m &=
  - \ii \lhatf \khat_z \Ehatp \left( \Ahat_{m+1} + \Ahat_{m-1} \right)
  - \DB \khat^2_m \Ahat_m
  + \ii \khat_z \phat_m,
\end{align}

\section{Real Formulation}

The linearized equations can be transformed in real form (i.e., no
explicit $\ii$ coefficients) by replacing the hatted variables
($\phat_m, \That_m$, etc) with tilded variables ($\ptil_m, \Ttil_m$,
etc), defined by the expressions
%
\begin{equation}
  \phat_m = \ii^m \ptil_m, \qquad
  \That_m = \ii^{m+1} \Ttil_m, \qquad
  \Chat_m = \ii^{m+1} \Ctil_m, \quad
  \Ahat_m = \ii^{m+1} \Atil_m.
\end{equation}
%
Substituting these into the linearized equations, and dividing through
by common factors of $\ii$, we obtain
%
\begin{align}
  \begin{split}
  \shat \ptil_{m} &=
  \frac{\lhatf \khat_z}{\khat^2_m} \Ehatp \left[
    {\color{red}-} \left( \lhatf^2 - \khat^2_{m+1} \right) \ptil_{m+1} 
    + \left( \lhatf^2 - \khat^2_{m-1} \right) \ptil_{m-1}
    \right]
  - \Prd \khat^2_m \ptil_m
  {\color{red}+} \Prd \frac{m \lhatf}{\khat_m^2} \left( \Ttil_m - \Ctil_m \right) \\
  & \qquad {\color{red}-} \HB \khat_z \Atil_m,
  \end{split} \\
  \shat \Ttil_m &=
  - \lhatf \khat_z \Ehatp \left( {\color{red}-} \Ttil_{m+1} + \Ttil_{m-1} \right)
  - \lhatf \khat_z \EhatT \left( {\color{red}-} \ptil_{m-1} - \ptil_{m+1} \right)
  - m \lhatf \ptil_m
  - \khat^2_m \Ttil_m \\
  \shat \Ctil_m &=
  - \lhatf \khat_z \Ehatp \left( {\color{red}-} \Ctil_{m+1} + \Ctil_{m-1} \right)
  - \lhatf \khat_z \EhatC \left( {\color{red}-} \ptil_{m-1} - \ptil_{m+1} \right)
  - m \lhatf \frac{\ptil_m}{\Rz}
  - \tau \khat^2_m \Ctil_m \\
  \shat \Atil_m &=
  - \lhatf \khat_z \Ehatp \left( {\color{red}-} \Atil_{m+1} + \Atil_{m-1} \right)
  - \DB \khat^2_m \Atil_m
  + \khat_z \ptil_m
\end{align}
%
All of the $\ii$ factors have now gone, and some signs have flipped
(I've highlighted these in red).

\section{Matrix Structure}

The real-coefficient equations above can be arranged into eigenproblem form as
%
\begin{equation}
  \Amat \uvec = \shat \uvec,
\end{equation}
%
Let's assume the eigenvector $\uvec$ has elements
%
\begin{equation}
  \uvec =
  \begin{bmatrix}
    \vvec_{-N} \\
    \vvec_{-N+1} \\
    \vdots \\
    \vvec_{N-1} \\
    \vvec_{N}
  \end{bmatrix},
\end{equation}
%
where the 4-element `sub-vector' $\vvec_{m}$ is defined by
%
\begin{equation}
  \vvec_{m} =
  \begin{bmatrix}
    \ptil_{m} \\
    \Ttil_{m} \\
    \Ctil_{m} \\
    \Atil_{m}
  \end{bmatrix}.
\end{equation}
%
With this ordering of the dependent variables, the system matrix
$\Amat$ has a block-tridiagonal structure:
%
\begin{equation}
  \Amat =
  \begin{bmatrix}
    \Bmat_{-N,-N} & \Bmat_{-N,-N+1} & & & \\
    \Bmat_{-N+1,-N} & \Bmat_{-N+1,-N+1} & \Bmat_{-N+1,-N+2} & & \\
    & \ddots & \ddots & \ddots & \\
    & & \Bmat_{N-1,N-2} & \Bmat_{N-1,N-1} & \Bmat_{N-1,N} \\
    & & & \Bmat_{N,N-1} & \Bmat_{N,N}
  \end{bmatrix}
\end{equation}
%
with $4\times4$ blocks given by
%
\begin{equation}
  \Bmat_{m,m-1} =
  \begin{bmatrix}
    \frac{\lhatf \khat_z}{\khat^2_m} \Ehatp \left( \lhatf^2 - \khat^2_{m-1} \right) &
    0 &
    0 &
    0 \\
    \lhatf \khat_z \EhatT &
    -\lhatf \khat_z \Ehatp &
    0 &
    0 \\
    \lhatf \khat_z \EhatC &
    0 & 
    -\lhatf \khat_z \Ehatp &
    0 \\
    0 &
    0 &
    0 &
    -\lhatf \khat_z \Ehatp
  \end{bmatrix},
\end{equation}
%
\begin{equation}
  \Bmat_{m,m} =
  \begin{bmatrix}
    -\Prd \khat^2_m &
    \Prd \frac{m \lhatf}{\khat^2_m} &
    -\Prd \frac{m \lhatf}{\khat^2_m} &
    -\HB \khat_z \\
    - m \lhatf &
    -\khat^2_m &
    0 &
    0 \\
    - \frac{m \lhatf}{\Rz} &
    0 &
    - \tau \khat^2_m &
    0 \\
    \khat_z &
    0 &
    0 &
    - \DB \khat^2_m
  \end{bmatrix},
\end{equation}
%
and
%
\begin{equation}
  \Bmat_{m,m+1} =
  \begin{bmatrix}
    -\frac{\lhatf \khat_z}{\khat^2_m} \Ehatp \left( \lhatf^2 - \khat^2_{m+1} \right) &
    0 &
    0 &
    0 \\
    \lhatf \khat_z \EhatT &
    \lhatf \khat_z \Ehatp &
    0 &
    0 \\
    \lhatf \khat_z \EhatC &
    0 &
    \lhatf \khat_z \Ehatp &
    0 \\
    0 &
    0 &
    0 &
    \lhatf \khat_z \Ehatp
  \end{bmatrix}.
\end{equation}
%

\section{Parity Splitting}

Adrian's notes demonstrate how the system of linearized equations can
be split into subsystems with opposite parities. Let's
follow this approach for the real-coefficient equations, by defining
%
\begin{multline}
  \ptil_m^{\pm} = \ii^m \left( \phat_m \pm \phat_{-m} \right), \qquad
  \Ttil_m^{\pm} = \ii^{m+1} \left( \That_m \pm \That_{-m} \right), \\
  \Ctil_m^{\pm} = \ii^{m+1} \left( \Chat_m \pm \Chat_{-m} \right), \qquad
  \Atil_m^{\pm} = \ii^{m+1} \left( \Ahat_m \pm \Ahat_{-m} \right),
\end{multline}
%
for $m > 0$. The $m=0$ case is a little special, in that
$\ptil_{0}^{+} = 2 \ptil_{0}$ while $\ptil^{-} = 0$ (and similarly for
the other three variables). With these definitions, the equations for
$m > 0$ become
%
\begin{align}
  \begin{split}
  \shat \ptil_{m}^{\pm} &=
  \frac{\lhatf \khat_z}{\khat^2_m} \Ehatp \left[
    - \left( \lhatf^2 - \khat^2_{m+1} \right) \ptil_{m+1}^{\pm} 
    + \left( \lhatf^2 - \khat^2_{m-1} \right) \ptil_{m-1}^{\pm}
    \right]
  - \Prd \khat^2_m \ptil_m^{\pm}
  + \Prd \frac{m \lhatf}{\khat_m^2} \left( \Ttil_m^{\mp} - \Ctil_m^{\mp} \right) \\
  & \qquad - \HB \khat_z \Atil_m^{\pm},
  \end{split} \\
  \shat \Ttil_m^{\pm} &=
  - \lhatf \khat_z \Ehatp \left( - \Ttil_{m+1}^{\pm} + \Ttil_{m-1}^{\pm} \right)
  - \lhatf \khat_z \EhatT \left( - \ptil_{m-1}^{\mp} - \ptil_{m+1}^{\mp} \right)
  - m \lhatf \ptil_m^{\mp}
  - \khat^2_m \Ttil_m^{\pm}, \\
  \shat \Ctil_m^{\pm} &=
  - \lhatf \khat_z \Ehatp \left( - \Ctil_{m+1}^{\pm} + \Ctil_{m-1}^{\pm} \right)
  - \lhatf \khat_z \EhatC \left( - \ptil_{m-1}^{\mp} - \ptil_{m+1}^{\mp} \right)
  - m \lhatf \frac{\ptil_m^{\mp}}{\Rz}
  - \tau \khat^2_m \Ctil_m^{\pm}, \\
  \shat \Atil_m^{\pm} &=
  - \lhatf \khat_z \Ehatp \left( - \Atil_{m+1}^{\pm} + \Atil_{m-1}^{\pm} \right)
  - \DB \khat^2_m \Atil_m^{\pm}
  + \khat_z \ptil_m^{\pm}.
\end{align}
%
For $m = 0$, the equations are
%
\begin{align}
  \shat \ptil_{0}^{+} &=
  2 \frac{\lhatf \khat_z}{\khat^2_0} \Ehatp \left[
    - \left( \lhatf^2 - \khat^2_{1} \right) \ptil_{1}^{+} 
    \right]
  - \Prd \khat^2_0 \ptil_0^{+}
  - \HB \khat_z \Atil_0^{+},
  \\ 
  \shat \Ttil_0^{+} &=
  - 2 \lhatf \khat_z \Ehatp \left( - \Ttil_{1}^{+} \right)
  - 2 \lhatf \khat_z \EhatT \left( - \ptil_{1}^{-} \right)
  - \khat^2_0 \Ttil_0^{+}, \\
  \shat \Ctil_0^{+} &=
  - 2 \lhatf \khat_z \Ehatp \left( - \Ctil_{1}^{+} \right)
  - 2 \lhatf \khat_z \EhatC \left( - \ptil_{1}^{-} \right)
  - \tau \khat^2_0 \Ctil_0^{+}, \\
  \shat \Atil_0^{+} &=
  - 2 \lhatf \khat_z \Ehatp \left( - \Atil_{1}^{+} \right)
  - \DB \khat^2_0 \Atil_0^{+}
  + \khat_z \ptil_0^{+}.
\end{align}
%
(these look slightly different from Adrian's notes, because we are
using $\ptil_{0}^{+} = 2 \ptil_{0}$ rather than $\ptil_{0}$, etc;
that's what causes the additional factors of two).

With this parity splitting, the system of equations splits into a pair
of decoupled subsystems: one involving only $\ptil_m^+$, $\Ttil_m^-$,
$\Ctil_m^-$ and $\Atil_m^+$, and the other only $\ptil_m^-$,
$\Ttil_m^+$, $\Ctil_m^+$ and $\Atil_m^-$. Similarly to before, each
subsystem can be arranged into eigenproblem form as
%
\begin{equation}
  \Amat^{\pm} \uvec^{\pm} = \shat \uvec^{\pm},
\end{equation}
%
Let's assume the eigenvectors $\uvec^{\pm}$ have elements
%
\begin{equation}
  \uvec^{\pm} =
  \begin{bmatrix}
    \vvec_{0}^{\pm} \\
    \vvec_{1}^{\pm} \\
    \vdots \\
    \vvec_{N-1}^{\pm} \\
    \vvec_{N}^{\pm}
  \end{bmatrix},
\end{equation}
%
where the subvectors $\vvec_{m}^{\pm}$ are defined by
%
\begin{equation}
  \vvec_{m}^{\pm} =
  \begin{bmatrix}
    \ptil_{m}^{\pm} \\
    \Ttil_{m}^{\mp} \\
    \Ctil_{m}^{\mp} \\
    \Atil_{m}^{\pm}
  \end{bmatrix}
\end{equation}
%
for $m > 0$. The $m = 0$ subvectors are special, in that they have a length of 2 rather than 4:
%
\begin{equation}
  \vvec_{0}^{+} =
  \begin{bmatrix}
    \ptil_{0}^{+} \\
    \Atil_{0}^{+}
  \end{bmatrix},
  \qquad
  \vvec_{0}^{-} =
  \begin{bmatrix}
    \Ttil_{0}^{-} \\
    \Ctil_{0}^{-}
  \end{bmatrix}.
\end{equation}
%
With this ordering of the dependent variables, the system matrices $\Amat^{\pm}$ take the form
%
\begin{equation}
  \Amat^{\pm} =
  \begin{bmatrix}
    \Bmat_{0,0}^{\pm} & \Bmat_{0,1}^{\pm} & & & \\
    \Bmat_{1,0}^{\pm} & \Bmat_{1,1}^{\pm} & \Bmat_{1,2}^{\pm} & & \\
    & \ddots & \ddots & \ddots & \\
    & & \Bmat_{N-1,N-2}^{\pm} & \Bmat_{N-1,N-1}^{\pm} & \Bmat_{N-1,N}^{\pm} \\
    & & & \Bmat_{N,N-1}^{\pm} & \Bmat_{N,N}^{\pm}
  \end{bmatrix}.
\end{equation}
%
The definitions of the blocks $\Bmat^{\pm}$ are the same as the
corresponding $\Bmat$, with the exception of the top-left corner:
%
\begin{equation}
  \Bmat_{0,0}^{+} =
  \begin{bmatrix}
    - \Prd \khat_0^2 &
    - \HB \khat_z \\
    \khat_z &
    - \DB \khat_0^2
  \end{bmatrix},
  \qquad
  \Bmat_{0,0}^{-} =
  \begin{bmatrix}
    - \khat_0^2 &
    0 \\
    0 &
    -\tau \khat_0^2
  \end{bmatrix},
\end{equation}
%
\begin{equation}
  \Bmat_{1,0}^{+} =
  \begin{bmatrix}
    \frac{\lhatf \khat_z}{\khat^2_1} \Ehatp \left( \lhatf^2 - \khat^2_{0} \right) &
    0 \\
    \lhatf \khat_z \EhatT &
    0 \\
    \lhatf \khat_z \EhatC &
    0 \\
    0 &
    -\lhatf \khat_z \Ehatp
  \end{bmatrix},
  \qquad
  \Bmat_{1,0}^{-} =
  \begin{bmatrix}
    0 &
    0 \\
    -\lhatf \khat_z \Ehatp &
    0 \\
    0 & 
    -\lhatf \khat_z \Ehatp \\
    0 &
    0
  \end{bmatrix},
\end{equation}
%
and
%
\begin{equation}
  \Bmat_{0,1}^{+} =
  \begin{bmatrix}
    - 2 \frac{\lhatf \khat_z}{\khat^2_0} \Ehatp \left( \lhatf^2 - \khat^2_{1} \right) &
    0 &
    0 &
    0 \\
    0 &
    0 &
    0 &
    2 \lhatf \khat_z \Ehatp
  \end{bmatrix},
  \qquad
  \Bmat_{0,1}^{-} =
  \begin{bmatrix}
    2 \lhatf \khat_z \EhatT &
    2 \lhatf \khat_z \Ehatp &
    0 &
    0 \\
    2 \lhatf \khat_z \EhatC &
    0 &
    2 \lhatf \khat_z \Ehatp &
    0
  \end{bmatrix}.
\end{equation}




\end{document}

  
    
    
  
  
    
  
  
    
